\documentclass[12pt,a4paper]{article}
%\documentclass[fleqn]{scrartcl}
\usepackage[english]{babel}
\usepackage{amsmath}
\usepackage{graphicx}
\usepackage{hyperref}
\usepackage{mathrsfs}  
%\usepackage[colorlinks=true,linkcolor=blue,urlcolor=blue,citecolor=blue,pdfusetitle]{hyperref}
\selectlanguage{english}
\usepackage{mathtools}
\DeclarePairedDelimiter\bra{\langle}{\rvert}
\DeclarePairedDelimiter\ket{\lvert}{\rangle}
\DeclarePairedDelimiterX\braket[2]{\langle}{\rangle}{#1 \delimsize\vert #2}
\usepackage{braket}
\usepackage{appendix}
\usepackage{ amssymb }
\usepackage{amsmath, amsthm}
\usepackage[backend=biber,style=ieee,citestyle=numeric-comp, url=false, eprint=false, url=true, isbn=false, doi=false]{biblatex}
\addbibresource{library.bib}
\begin{document}
\begin{titlepage}
\begin{center}
\noindent\rule{\textwidth}{1pt}
\Large\textbf{Lindblad}\\
\noindent\rule{\textwidth}{1pt}
\vspace{0.5cm}
\large{Master-Projekt}\\
%\large{nature12016}\\
\vspace{3cm}
%\normalsize{H. Bernien1, B. Hensen1, W. Pfaff1, G. Koolstra1, M. S. Blok1, L. Robledo1, T. H. Taminiau1, M. Markham2, D. J. Twitchen2,L. Childress3 R.Hanson}\\
\vspace{2cm}

\normalsize{ Sander Stammbach}\\
\normalsize{Prof. Patrick Plotts}\\
\vspace{0.2cm}
12 November 2022\\
\end{center}

\nopagebreak
\vspace{1cm}
\begin{minipage}{.25\textwidth}
 \begin{flushleft}
 \end{flushleft}
\end{minipage}
\hskip.4\textwidth
\begin{minipage}{.25\textwidth}
\end{minipage}
\end{titlepage}
\tableofcontents
\newpage
\section{Introduction}

One of the most important question in thermodynamics is, how to convert thermal energy into work. For such tasks it exist many classical engines, as example the steam-machine. To quantify heat-engines, its commen to look at the ergotropy. In my master-project I will quantify the a three level maser. The three-level maser is a Quantum heat engine (short: QHE).
The work extraction from a classicle heat engine is a moving piston. But in this case it is a driving field.
Experimentally we need two different reservoir. The high-temperature reservoir can be realized by a gas noise lamp and the filter by a wave guide cutting off the lower frequencies.
 (Quelle main Paper)

\section{Theory}
\subsection{3-level-Maser}
In a three level system you have the three energy levels E1<E2<E3. Pumping is from the lowest level to the highest level E3. The condition for the third level is that it falls to the middle level E2 very quickly. On average, the system is almost not in the third state. The second system should then have a higher decay time, so that a population inversion can build up. This means that several particles are in the energetically higher state. From this state they come almost exclusively through stimulated emission into the deeper system E1. Stimulated emission is a necessary condition for coherent light. Coherent light Means they have the same phase and same frequency. (Quelle Wiki)
In the case of a normal laser the population inversion is achieved through a pumping light.
In the case of this calculation the higher level will be reached with a interaction of a warm bath.%do witer mache
\subsection{phase-averaged coherent states (PHAV)}
The output of a Laser is coherent light.
The quantum description of coherent light is a coherent state. The photon number distribution of coherent light is a poisson distribution. 
The Pignerfunction from a coherent state itself is a Gaussian. But the Wignerfunction of a pahse-average-state has a non-Gaussian Wignerfunction. 
The mathematical description of a  PHAV is the same as a normal coherent state but with a random phase. So we get a new term of $exp(i\pi\phi)$ in it. 
The PHAV state could represented by following formula (Quelle 5):
\begin{equation}
\ket{\alpha}=e^{-1/2|\alpha|}\sum_{n=0}^{\infty}\frac{|\alpha|^n e^{in\phi}}{\sqrt{n!}}
\end{equation}
\subsection{Lindblad-Master-Equation}
The Basic of our system is a three-level quantum system. This three level system is driven by a hot bath and a cold bath. Those have the temperature $T_c$ and $T_h$. This quantum system lives in a cavity. 
In my calculation, the thermal bath is constant, so I can use the Lindblad-masterequation. 
In the following figure is shown a three-level system:
The efficiency of the maser is given by the formula:
\begin{equation}
\eta_{maser}=\frac{\omega_f}{\omega_h}
\end{equation}

\newpage
The master equation is:
\begin{equation}
\dot{\rho}(t)=\frac{1}{i \hbar}[H,\rho]+ \mathcal{L}\rho
\end{equation}
The first part of formulais (1) is the von Neuman-equation. The von Neuman-equation is the analog of the Schrödinger-equation but for density matrices. This part of the equation is unitary and therefore the process is reversible.
The non-unitary part of the equation 
$\mathcal{L}$ have three parts. $\mathcal{L}_h$describe the interaction with the hot bath.
$\mathcal{L}_c$ Describe the interaction with the cold bath.
$\mathcal{L}_{cav}$Describe the Photons which leaves the cavity. so if we have a small $\kappa$ means less photons will leave and stay in the cavity. we see that in the Fockplott.

The interaction Hamiltonian or Jaynes-Cummings Hamiltonian is:
\begin{equation}
H_{int}=\hbar g(\sigma_{12}a^{\dag}+\sigma_{21}a)
\end{equation}
The  Hamiltonian of the photons is, which describes the phonons in the cavity:
\begin{equation}
H_{free}=\sum_{i=1}^3 \hbar \omega_i \ket{i}\bra{i}+\hbar \omega_f a^{\dag}a
\end{equation}
The total Hamiltonian
\begin{equation}
H=H_{free}+H_{int}
\end{equation}
\newpage
The interaction with the various environmental heat baths is described by the Liouvillian:
\begin{equation}
\mathcal{L}\hat{\rho}=\frac{\gamma_h}{2}\bar{n}(\omega_h,T_h)+1   \cdot \mathcal{D}[\sigma_{13}]
+\frac{\gamma_h}{2}\bar{n}(\omega_h,T_h)\cdot \mathcal{D}[\sigma_{31}]$$\\$$
+\frac{\gamma_c}{2}\bar{n}(\omega_c,T_c)+1)\cdot \mathcal{D}[\sigma_{23}]
+\frac{\gamma_c}{2}\bar{n}(\omega_c,T_c) \cdot	 \mathcal{D}[\sigma{32}]$$\\$$
\kappa\bar{n}(\omega_f,Tf)+1	\cdot\mathcal{D}[a]+
\kappa \bar{n}(\omega_f,T_f)\cdot \mathcal{D}[a^{\dag}]
\end{equation}

$\mathcal{D}$ is defined with this formula:
\begin{equation}
\mathcal{D}[A]=(2A \rho	A^{\dag}-A^{\dag}A\rho-\rho A^{\dag}A)
\end{equation}

The Bose-Einstein statistic is a probability distribution in quantum statistics . It describes the mean occupation number $\langle n(E) \rangle$ of a quantum state of energy $E$, in thermodynamic equilibrium at absolute temperature $T $ for identical bosons as occupying particles. n depends on the temperature and the frequency.
n is defined as:

\begin{equation}
n(\omega,T)=\frac{1}{\exp[\frac{\hbar \omega_i}{k_b T_i}]-1}
\end{equation}
The  prefactor $\gamma_i$.....................
The Liouvillian have different constants. The coupling constants $g$ for the Hamiltonian and the $\kappa$ for the Liouvillian part. 





\subsection{Probability calculation}
 When we work with density matrices, its commen to work with expectation values with $<A>=Tr[A*\rho]$.
 $A$ is a operator and describe a measurement.
With this we can calculate the probabilty 
To calculate the expected heat flow we can take the partial trace from 
\begin{equation}
<J>=Tr[\rho_{free}\cdot \mathcal{L}_h[\rho]]+Tr[\rho_{free}\cdot \mathcal{L}_c[\rho]]+Tr[\rho_{free}\cdot \mathcal{L}_{cav}[\rho]]
\end{equation}


\section{Calculation}
\subsection{Software}
For the hole implementation of the tree-level-system in a cavity, I used qutip. Qutip is library in python, which allows to solve Masterequation pretty fast.
\subsection{implementation of the tree-levelsystem in qutip}
In our case only $\omega_f$ interact with the light. 
first i defined the frequencies $ \omega_c $, $ \omega_h$ and $ \omega_f$ 
The constants $h $ and the bolzmanfactor $k_b$ are 1.
alsou defined as constants are the three different Boseeinstein-distributions $n_h$, $n_c$ and $n_f$
The transition-operators are  made by following qutip implementation: "Trans13=tensor(vg*va.dag(),qutip.identity(nph))". nph is the maximum of the photonnumber in the cavity. so I get 90 x 90 matrices. 
similarly the projectors. 
with those its easy to construct the hamiltoniens, $H_{free}$ and $H_{int}$ as in formula 3 and 4.
To calculate the the density matrices for steadystates we can alsou use a qutip function, call steadystate().
this function needs the total Hamiltonian and a list of the non-unitary operators as arguments.
We can we can construct this list as a multiplication of our transition-operators and the tree different Boseeinstein-distributions times the different $\gamma$-factors. 
as output of the function steady state we get the density-matrices for steady-states.
\subsection{Out}
First i made a Fockplot and a wigner function of the reduce density matrices $\rho_{free}$.
Because $\rho=\rho_{atom}\otimes \rho_{free}$, I can make the partial trace of $\rho$ with qutip, to trace out the reduced density matrices $\rho_free$. The Fock-plot and the Wigner-plot is also done with a qutip function.
with the density matrix times the $L_{p}\rho$ i calculated on the heat flux by taking the trace of $H \mathcal{L}\cdot \rho$. 
and plot this for 200 different  $g$ `s 
so the goal of this work is to find Einstein-Bose-distributions which yield a RAHV state.
as in the paper (Quelle2) 



\section{Results}



\section{Discussion}

\section{References}






\begin{equation}
\mathcal{L}\hat{\rho}=\frac{\gamma_h}{2}\biggl[  \frac{1}{\exp[\frac{\hbar \omega_h}{k_b T_h}]-1}+1   \biggr]\cdot \biggl(2 \sigma_{13}\cdot\rho\cdot \sigma_{13}^{\dag}-\sigma_{13}^{\dag}\sigma_{13}\rho-\rho\sigma_{13}^{\dag}\sigma_{13}\biggr) $$\\$$
+\frac{\gamma_h}{2}\bigg[  \frac{1}{\exp[\frac{\hbar \omega_h}{k_b T_H}]-1}\biggr] \cdot\biggl( 2 \sigma_{31}\cdot\rho\cdot \sigma_{31}^{\dag} -\sigma_{31}^{\dag}\sigma_{31}\rho-\rho\sigma_{31}^{\dag}\sigma_{31}\biggr)$$\\$$
+\frac{\gamma_c}{2}\biggl[  \frac{1}{\exp[\frac{\hbar \omega_c}{k_b T_c}]-1}+1   \biggr]\cdot \biggl(2 \sigma_{23}\cdot\rho\cdot \sigma_{23}^{\dag}-\delta_{23}^{\dag}\sigma_{23}\rho-\rho\sigma_{23}^{\dag}\sigma_{23}\biggr)$$\\$$
+\frac{\gamma_c}{2}\bigg[  \frac{1}{\exp[\frac{\hbar \omega_c}{k_b T_c}]-1}\biggr]
\cdot\biggl( 2 \sigma_{32}\cdot\rho\cdot \sigma_{32}^{\dag}-\sigma_{32}^{\dag}\sigma_{32}\rho-\rho\sigma_{32}^{\dag}\sigma_{32} \biggr)$$\\$$
\kappa\biggl[ \frac{1}{\exp[\frac{\hbar w_f}{k_b T_f}]-1}+1\biggr] \cdot\biggl( 2 a\rho a^{\dag} -a^{\dag}a\rho-\rho a^{\dag}a\biggr)$$\\$$
\kappa\biggl[ \frac{1}{\exp[\frac{\hbar w_f}{k_b T_f}]-1}\biggr]\cdot \biggl(2 a^{\dag}\rho a -aa^{\dag}\rho-\rho a a^{\dag}\biggr)$$\\$$
\end{equation}

\end{document}
