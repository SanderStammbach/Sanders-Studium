\documentclass[12pt,a4paper]{article}
%\documentclass[fleqn]{scrartcl}
\usepackage[english]{babel}
\usepackage{amsmath}
\usepackage{graphicx}
\usepackage{hyperref}
\usepackage{mathrsfs}  
%\usepackage[colorlinks=true,linkcolor=blue,urlcolor=blue,citecolor=blue,pdfusetitle]{hyperref}
\selectlanguage{english}
\usepackage{mathtools}
\DeclarePairedDelimiter\bra{\langle}{\rvert}
\DeclarePairedDelimiter\ket{\lvert}{\rangle}
\DeclarePairedDelimiterX\braket[2]{\langle}{\rangle}{#1 \delimsize\vert #2}
\usepackage{braket}
\usepackage{appendix}
\usepackage{ amssymb }
\usepackage{amsmath, amsthm}
\begin{document}
\begin{titlepage}
\begin{center}
\noindent\rule{\textwidth}{1pt}
\Large\textbf{Lindblad}\\
\noindent\rule{\textwidth}{1pt}
\vspace{0.5cm}
\large{Projekt}\\
\large{nature12016}\\
\vspace{3cm}
%\normalsize{H. Bernien1, B. Hensen1, W. Pfaff1, G. Koolstra1, M. S. Blok1, L. Robledo1, T. H. Taminiau1, M. Markham2, D. J. Twitchen2,L. Childress3 R.Hanson}\\
\vspace{2cm}

\normalsize{ Sander Stammbach}\\
\normalsize{Prof. Patrick Plotts}\\
\vspace{0.2cm}
12 Oktober 2021\\
\end{center}

\nopagebreak
\vspace{1cm}
\begin{minipage}{.25\textwidth}
 \begin{flushleft}
 \end{flushleft}
\end{minipage}
\hskip.4\textwidth
\begin{minipage}{.25\textwidth}
\end{minipage}
\end{titlepage}
\tableofcontents
\newpage
\section{Introduction}

The most important question in thermodynamics is, how to convert thermal energy into work. For such tasks it exist many classical engines, as example the steam-machine. To quantify heat-engines, its commen to look at the ergotropy. In my master-project I will quantify the a three level maser. The three-level maser is a Quantum heat engine (short: QHE).
The work extraction from a classicle heat engine is a moving piston. But in this case it is a driving field..
Experimentally we need two different reservoir. The high-temperature reservoir can be realized by a gas noise lamp and the filter by a wave guide cutting off the lower frequencies.
 

\section{Theory}
\subsection{Model}
The Basic of our system is a three-level quantum system. This three level system is driven by a hot bath and a cold bath. Those have the temperature $T_c$ and $T_h$. This quantum system lives in a cavity. 
In my calculation, the thermal bath is constant, so i can use the Lindblad-masterequation. 
The efficiency is given by the formula:
\begin{equation}
\eta_M=\frac{\nu_s}{\nu_p}
\end{equation}

\newpage
The master equation is:
\begin{equation}
\dot{\rho}(t)=\frac{1}{i \hbar}[H,\rho]+ \mathcal{L}\rho
\end{equation}
The first part is the vonNeuman equation. The von Neumanequation is the analog of the Schrödinger-equation but for density matrices. This part of the equation is unitary and therefore the process is reversible.
The non-unitary part of the equation 
$\mathcal{L}$ have three parts. $\mathcal{L}_h$describe the interaction with the hot bath.
$\mathcal{L}_c$ Describe the interaction with the cold bath.
$\mathcal{L}_{cav}$Describe the Photons which leaves the cavity. so if we have a small $\kappa$ means less photons will leave and stay in the cavity. we see that in the Fockplott.

The interaction Hamiltonian or Jaynes-Cummings Hamiltonian is:
\begin{equation}
H_{int}=\hbar g(\sigma_{12}a^{\dag}+\sigma_{21}a)
\end{equation}
The  Hamiltonian of the photons is, which describes the phonons in the cavity:
\begin{equation}
H_{free}=\sum_{i=1}^3 \hbar \omega_i \ket{i}\bra{i}+\hbar \omega_f a^{\dag}a
\end{equation}
The total Hamiltonian
\begin{equation}
H=H_{free}+H_{int}
\end{equation}
The interaction with the various environmental heat baths is described by the Liouvillian:
\begin{equation}
\mathcal{L}\hat{\rho}=\frac{\gamma_h}{2}\biggl[  \frac{1}{\exp[\frac{\hbar \omega_h}{k_b T_h}]-1}+1   \biggr]\cdot \biggl(2 \sigma_{13}\cdot\rho\cdot \sigma_{13}^{\dag}-\sigma_{13}^{\dag}\sigma_{13}\rho-\rho\sigma_{13}^{\dag}\sigma_{13}\biggr) $$\\$$
+\frac{\gamma_h}{2}\bigg[  \frac{1}{\exp[\frac{\hbar \omega_h}{k_b T_H}]-1}\biggr] \cdot\biggl( 2 \sigma_{31}\cdot\rho\cdot \sigma_{31}^{\dag} -\sigma_{31}^{\dag}\sigma_{31}\rho-\rho\sigma_{31}^{\dag}\sigma_{31}\biggr)$$\\$$
+\frac{\gamma_c}{2}\biggl[  \frac{1}{\exp[\frac{\hbar \omega_c}{k_b T_c}]-1}+1   \biggr]\cdot \biggl(2 \sigma_{23}\cdot\rho\cdot \sigma_{23}^{\dag}-\delta_{23}^{\dag}\sigma_{23}\rho-\rho\sigma_{23}^{\dag}\sigma_{23}\biggr)$$\\$$
+\frac{\gamma_c}{2}\bigg[  \frac{1}{\exp[\frac{\hbar \omega_c}{k_b T_c}]-1}\biggr]
\cdot\biggl( 2 \sigma_{32}\cdot\rho\cdot \sigma_{32}^{\dag}-\sigma_{32}^{\dag}\sigma_{32}\rho-\rho\sigma_{32}^{\dag}\sigma_{32} \biggr)$$\\$$
\kappa\biggl[ \frac{1}{\exp[\frac{\hbar w_f}{k_b T_f}]-1}+1\biggr] \cdot\biggl( 2 a\rho a^{\dag} -a^{\dag}a\rho-\rho a^{\dag}a\biggr)$$\\$$
\kappa\biggl[ \frac{1}{\exp[\frac{\hbar w_f}{k_b T_f}]-1}\biggr]\cdot \biggl(2 a^{\dag}\rho a -aa^{\dag}\rho-\rho a a^{\dag}\biggr)$$\\$$
\end{equation}

 When we work with density matrices, its commen Since we are working with density matrices here, it is easy to work with expectation values with $<A>=Tr[A*\rho]$.


\section{Calculation}
In our case only $\omega_f$ interact with the light.

\subsection{Software}
For the hole implementation of the tree-level-system in a cavity I used qutip. Qutip is library in python, which allows to solve Masterequation pretty fast.










\end{document}
